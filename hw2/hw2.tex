\documentclass[a4paper,10pt]{article}
\usepackage[utf8]{inputenc}
\usepackage{amssymb}
\usepackage{amsfonts}
\usepackage{amsmath}
\usepackage{enumerate}
\setlength{\parindent}{0pt}
\usepackage[margin=1in]{geometry}

\usepackage{listings}
\usepackage{color}

\definecolor{dkgreen}{rgb}{0,0.6,0}
\definecolor{gray}{rgb}{0.5,0.5,0.5}
\definecolor{mauve}{rgb}{0.58,0,0.82}

\lstset{frame=tb,
  language=C++,
  aboveskip=3mm,
  belowskip=3mm,
  showstringspaces=false,
  columns=flexible,
  basicstyle={\small\ttfamily},
  numbers=none,
  numberstyle=\tiny\color{gray},
  keywordstyle=\color{blue},
  commentstyle=\color{dkgreen},
  stringstyle=\color{mauve},
  breaklines=true,
  breakatwhitespace=true,
  tabsize=3
}
\begin{document}

Jason Qiu

CS 385 Homework Assignment $\#2$

\emph{I pledge my honor that I have abided by the Stevens Honor System.}

\section*{p. 67, $\#4$ a-e}
\begin{enumerate}[(a)]
\item This algorithm computes $\sum_{i = 1}^{n}i^2$
\item Its basic operation is \verb|S <- S + i * i|.
\item The operation is performed $n$ times.
\item The efficiency class of the algorithm is $\Theta(n)$.
\item Using the fact that $\sum_{i=1}^{n} i^2 = \frac{n(n+1)(2n+1)}{6}$, the following algorithm that runs in $\Theta(1)$ can be used instead:

\begin{lstlisting}
int mystery(int n){
	return n*(n+1)*(2n+1)/6;
}
\end{lstlisting}
\end{enumerate}

\section*{p. 76, $\#1$ a-e}
\begin{enumerate}[(a)]
\item \begin{enumerate}[Step 1.]
	\item $x(n-1) = x(n-2) + 5$
	
	$x(n) = x(n-1) + 5 = x(n-2) + 2(5)$
	\item $x(n-2) = x(n-3) + 5$
	
	$x(n) = x(n-2) + 2(5) = x(n-3) + 3(5)$
	\item $x(n) = x(n-i) + i(5)$
	
	\item Initial condition $x(1) = 0$, so $n-i = 1 \Rightarrow i = n-1$
	
	\item $x(n) = x(1) + (n-1)(5) = \boxed{5n-5}$
\end{enumerate}

\item \begin{enumerate}[Step 1.]
	\item $x(n-1) = 3x(n-2)$
	
	$x(n) = 3x(n-1) = 3(3(x(n-2)) = 3^2x(n-2)$
	\item $x(n-2) = 3x(n-3)$
	
	$x(n) = 3^2x(n-2) = 3^2(3x(n-3)) = 3^3x(n-3)$
	\item $x(n) = 3^ix(n-i)$
	
	\item Initial condition $x(1) = 4$, so $n-i = 1 \Rightarrow i = n-1$
	
	\item $x(n) = 3^{n-1}x(1) = \boxed{4\cdot 3^{n-1}}$
\end{enumerate}

\item \begin{enumerate}[Step 1.]
	\item $x(n-1) = x(n-2) + n-1$
	
	$x(n) = x(n-1) + n = x(n-2) + (n-1) + n$
	\item $x(n-2) = x(n-3) + n-2$
	
	$x(n) = x(n-2) + (n-1) + n  = x(n-3) + (n-2) + (n-1) + n$
	\item $x(n) = x(n-i) + (n-i+1) + (n-i+2) + \dots + n$
	
	\item Initial condition $x(0) = 0$, so $n-i = 0 \Rightarrow i = n$
	
	\item $x(n) = x(0) + 1 + 2 + \dots + n-1 + n = \boxed{\frac{n(n+1)}{2}}$
\end{enumerate}

\item \begin{enumerate}[Step 1.]
	\item $x(\frac{n}{2}) = x(\frac{x}{4}) + \frac{n}{2}$
	
	$x(n) = x(\frac{n}{2}) + n = x(\frac{x}{4}) + \frac{n}{2} + n$
	\item $x(\frac{n}{4}) = x(\frac{n}{8}) + \frac{n}{4}$
	
	$x(n) = x(\frac{n}{4}) + \frac{n}{2} + n  = x(\frac{n}{8}) + \frac{n}{4} + + \frac{n}{2} + n$
	\item $x(n) = x(\frac{n}{2^k}) + \frac{n}{2^{k-1}} + \frac{n}{2^{k-2}} + \dots + n$
	
	\item Initial condition $x(1) = 1$, so $n = 2^k \Rightarrow k = \text{lg}(n)$
	
	\item $x(n) = x(1) + \frac{n}{2^{\text{lg}(n) - 1}} +  \frac{n}{2^{\text{lg}(n) - 2}}  + \dots + n = \sum_{i=0}^{k} 2^i = 2^{k+1} - 1 = 2^{\text{lg}(n)+1} - 1 = \boxed{2n-1}$
\end{enumerate}

\item \begin{enumerate}[Step 1.]
	\item $x(\frac{n}{3}) = x(\frac{n}{3^2}) + 1$
	
	$x(n) = x(\frac{n}{3}) + 1 = x(\frac{n}{3^2}) + 2$
	\item $x(\frac{n}{3^2}) = x(\frac{n}{3^3}) + 1$
	
	$x(n) = x(\frac{n}{3^2}) + 2  = x(\frac{n}{3^3}) + 3$
	\item $x(n) = x(\frac{n}{3^k}) + k$
	
	\item Initial condition $x(1) = 1$, so $n = 3^k \Rightarrow k = \log_3(n)$
	
	\item $x(n) = x(1) + k = \boxed{\log_3(n)+1}$
\end{enumerate}
\end{enumerate}

\section*{p. 76-77, $\#3$ a-b}
\begin{enumerate}[(a)]
\item \verb|S(n)| will run once plus however many times \verb|S(n-1)| runs. That is, $T(n) = T(n-1) + 1$ and $T(1) = 1$. Next, we solve the recurrence relation:

	\begin{enumerate}[Step 1.]
		\item $T(n-1) = T(n-2) + 1$
	
		$T(n) = T(n-1) + 1 = T(n-2)) + 2$
		\item $T(n-2) = T(n-3) + 1$
	
		$T(n) = T(n-2) + 2  = T(n-3) + 3$
		\item $T(n) = T(n-i) + i$
	
		\item Initial condition $T(1) = 1$, so $n - i = 1 \Rightarrow i = n-1$
	
		\item $T(n) = T(1) + n-1 = \boxed{n}$
	\end{enumerate}
\item The straightforward nonrecursive algorithm also uses $n$ iterations. (There also exists a formula for this sum that runs in constant time, but the word 'straightforward' leads me to assume that the algorithm in question is simply computing the sum iteratively).
\end{enumerate}
\end{document}